%%%%%%%%%%%%%%%%%%%%%%%%%%%%%%%%%%%%%%%%%%%%%%%%%%%%%%%%%%%%%%%%%%%%%%%%%%%%%%
\chapter{AlphaFlow architecture}
%%%%%%%%%%%%%%%%%%%%%%%%%%%%%%%%%%%%%%%%%%%%%%%%%%%%%%%%%%%%%%%%%%%%%%%%%%%%%%

\section{Work items}

Work items represent activities that 
When an activity has to be performed, a work item is created. Depending on the activity,
different work items 

Different work items are available to support various operations:

\begin{itemize}
    \item \code{task}s that are assigned to users (e.g. writing a document)
    \item \code{decision}s that need to be made (e.g. deciding if a written document should be published or not)
    \item \code{email}s that need to be send to the users assigned for another work item
\end{itemize}

Within a workflow, operations are typically followed by other operations. Work
items are therefore responsible to generate other work items:

\begin{itemize}
    \item \code{Task} work items create other work items when completed. (e.g. deciding if a changed document should be published)
    \item \code{Decision} work items create different work items depending on whether the decision was accepted or rejected. (e.g either start a \code{task} to edit a document or start a different work item to publish it)
    \item The \code{email} work item first generates a following \code{task} and then sends out \code{email}s for the responsible users.
\end{itemize}

Work items can roughly be categorized:

\begin{description}

    \item[Assignable work items] are assigned to one or more users and map some
        real world task. The exist for a longer time until the real world task
        is completed.

    \item[Automatic work items] are used to perform actions within the system
        and are performed automatically and immediately. The do not necessarily
        map real world tasks. Examples are work items implementing control flow
        (\code{switch}), code execution (\code{expression}) and other packaged functionality
        (\code{permission}, \code{alarm}, \code{email} \ldots).

    \item[Daemon work items] are specially used to support other work items
        within the workflow.  to support other work items. An example of a
        daemon work item is the \code{recursion} work item which replicates
        given work items for every sub-object of the content-object.
\end{description}

As AlphaFlow is tightly integrated in Zope and Plone, there are some work items
that are special to Zope and Plone. E.g. the \code{permission} work item allows
to change the permission settings of the content object and the
\code{dcworkflow} work item allows to emulate the DCWorkFlow state variable.

\subsection{Work item states}

Work items have a simple life cycle. It begins with the state ``active'' and
usually ends with ``complete''. In special situations the work item might
assume the ``failed'' or ``terminated'' state, either automatically or by
manual intervention.

  \begin{description}

    \item[active] means that the work item is beeing processed by the system.
        If it is an assignable work item it is waiting for feedback from the
        user. ``Daemon'' work items are waiting for signals from other work
        items. There can be multiple active work items at the same time in a
        workflow instance.

    \item[failed] means that some kind of exceptional state was detected while
        the work item was carried out. This might either be a common Python
        exception or an obvious inconsistency in the workflow or the
        application. The work item then needs manual care by an administrator.

    \item[complete] means that the activity the work item was supposed to
        perform was completed.

    \item[terminated] means that the work item was terminated manually by an
        administrator. This is used to put a workflow into a consistent state
        after a failure occured.

  \end{description}

\section{Workflow instances}

When a workflow runs on a content object, the work items are stored in a
``workflow instance'' (or ``process instance''). A workflow instance also
stores additional information about a single run of a workflow for one certain
content object, including which workflow is run, when it was started and
finished, and what state it is in.

Assigning a workflow to a certain content object creates a workflow instance.
Starting a workflow instance generates one or more initial work items. After
that those work items continue to control the workflow by generating more work
items. (See section ``work items'')

When only work items in ``complete'' or ``terminated'' state are left, the
workflow instance is considered to be completed. If only ``daemon'' work items
are left, the workflow instance will be considered complete as well.  After a
workflow instance is completed, the content object can be used for another
workflow.

\begin{notice}
    \begin{itemize}
    \item It is not possible to run multiple workflows at the same time for a single content object.
    \item Creating and starting a workflow instance can be done at once or seperately.
    \end{itemize}
\end{notice}

\subsection{States of a workflow instance}

Workflow instances have a life cycle, just as work items, with the additional
state: ``initiated''. This state allows to distinguish between assigning a
workflow to a content object and starting the workflow. The typical life cycle
for a workflow instance is: ``initiated'', ``active'' and ``complete''. In special
situations, just like work items, a workflow instance might assume the states
``failed'' or ``terminated''.  

  \begin{description}

      \item[initiated] means that the workflow instance has been created and
          was assigned to a content object but has not been started yet.

      \item[active] that the workflow instance was started and is currently
          running under normal conditions.

      \item[failed] means that at least one work item changed its state to
          ``failed'' and needs manual care. A workflow instance in this state
          is not usable by the application. Any further action by the work
          items will be blocked until the ``failed'' state was resolved.

      \item[complete] means that all of the work items are completed and the
          process finished without problems. 
          New workflows can be started for
          the content object.

      \item[terminated] means that the process instance was stopped prematurely
          by an administrator. This state is typically used when a ``failed''
          situation can not be resolved cleanly and the process is either terminated
          or restarted.

          When a process is terminated, all work items in ``active'' or
          ``failed'' state are terminated as well.

          New workflows can be started for the content object.

  \end{description}

\section{Activities}

How work items perform operations in a workflow is controlled by ``activities''.
Exactly one activity is associated with each work item.

The activity type determines what kind of operation a work
item performs: whether it represents a task to be handled by a user,
orchestrates the making of a decision, or sends out email notifications.

Besides having a type, activities can be configured. The configuration of an
activity controls the details of how associated work items perform their
operation. The subject line of messages sent by an email work item is given by
the ``mailSubject'' property of the work item's associated email activity.

Work items drive the workflow by creating more work items. Activities are
important for the creation of new work items in two ways:

\begin{enumerate}
    \item The activity type determines \emph{when} to create new work items.
    \item The activity's configuration specifies \emph{what} work items are created.
\end{enumerate}

For example, a task work item can create other work items both when it is
created itself, and when it is declared completed. An email work item creates
other work items upon creation and before it sends out notifications. For each
time a work item needs to create other work items, its activity has a
configuration property that lists the activities to create new work items for.

\section{Workflow definitions}

Just as workflow instances collect work items, workflow definitions collect
activities. Each workflow instance is associated with exactly one workflow
definition.

Workflow definitions have some configuration properties of their own. Most
importantly, they have a list of start-up activities. When a workflow instance
is started, it creates work items for each start-up activity of its workflow
definition to get the workflow running.

A workflow definition may contain more than one activity of any given type,
each with its own configuration. Activities have an ID to distinguish them
within a workflow definition. It is by their ID that activities refer to each
other, for example when listing completion activities. Workflow definitions
list their start-up activities by their IDs as well.

Activities can only refer to other activities within the same workflow
definition.  Thus, a workflow definition fully describes what is possible in
any associated workflow instance.

\section{Instance configuration}

Work items determine the details of their operation by looking up how their
associated activities are configured. The same activity may be associated with
work items in several workflow instances, running on different content
objects. However, it is not always desirable to share the activity's entire
configuration between content objects. Therefore, activity properties may be
customized per workflow instance.

In such cases, the activity configuration expresses that the work item should
look up a certain property with its own workflow instance. For example, a task
activity ``edit'' might only list possible assignees. Each instance of the
workflow containing ``edit'' then has a property that lists some of them as
actual assignees. Any ``edit'' work item will be assigned to the actual
assignees for ``edit'' as listed by the workflow instance it belongs to.

Instance configuration is done by configuration work items. They may be run at
any point of a workflow. A configuration work item prompts an assigned user
for per-instance properties of one or more activities and stores them on its
own workflow instance.

It is entirely possible to change a workflow instance's configuration of
activities while they have active work items in that instance. If multiple
work items of the same activity are running simultaneously in a given workflow
instance, they share any per-instance configuration of their activity.
