%%%%%%%%%%%%%%%%%%%%%%%%%%%%%%%%%%%%%%%%%%%%%%%%%%%%%%%%%%%%%%%%%%%%%%%%%%%%%%
\chapter{Administering AlphaFlow}
%%%%%%%%%%%%%%%%%%%%%%%%%%%%%%%%%%%%%%%%%%%%%%%%%%%%%%%%%%%%%%%%%%%%%%%%%%%%%%

This chapter describes AlphaFlow's administration facilities in the Zope
Management Interface (ZMI). You find them in the \member{workflow\_manager} 
in your Plone site after installing AlphaFlow.

There are two aspects to administering AlphaFlow: managing workflow
definitions, and managing workflow instances.

%%%%%%%%%%%%%%%%%%%%%%%%%%%%%%%%%%%%%%%%%%%%%%%%%%%%%%%%%%%%%%%%%%%%%%%%%%%%%%
\section{Definitions}

The definitions view is a list of all AlphaFlow workflow definitions that
exist in the Plone portal. For each definition, the list displays its id,
title, and description and has actions for editing and deleting. 

The ``edit'' action will start the graphical workflow editor. \notice{The
graphical editor is only a technology preview. It is not even in alpha state.
The preferred way to create and edit workflows is to use ALF files.}

Using the definitions view  you can also import workflow definitions by
uploading ALF files from your computer. The chapter ``activity reference''
shows how to use ALF files.

You can enter an \var{id} by which the workflow definition will be known. It
must not yet be used by another AlphaFlow workflow definition. If you omit the
\var{id}, it is taken from ALF file.

%%%%%%%%%%%%%%%%%%%%%%%%%%%%%%%%%%%%%%%%%%%%%%%%%%%%%%%%%%%%%%%%%%%%%%%%%%%%%%
\section{Instances}

The ``instances'' view of the \member{workflow\_manager} offers an overview of
all workflow instances that are or were running in the portal.

Initially no instances will be shown. You can select different filters
based on the workflow instances states to get a list of all process instances
in that state.

Each entry in the instances list tells you the state of the instance and what
work items are currently active. Furthermore, it contains links to the
instance's overview, its workflow definition, and its content object.

\notice{Beware: Showing \emph{all} instances can be very slow and time
consuming on large installations. Use one of the filtered lists for better
performance.}

\subsection{Controlling a workflow instance}

Clicking on an instance in the instance list brings you to the overview page of
the workflow instance.  This page displays the instance's state, workflow
definition, and content object. It also lists all its work items, and displays
the event log.

The following actions can be performed on a workflow instance:
\begin{description}
\item[start] For an ``initiated'' instance, create work items for all start-up
  activities and change state to ``active''.
\item[dropin] Put a fall-out instance back in ``active'' state if there are
  currently no fall-out work items left.
\item[terminate] Terminate all active work items of the instance and change
  state to ``terminated''.
\item[reset] Terminate all active work items of the instance and change state
  to ``initiated''.
\item[restart] Terminate all active work items of the instance, create work
  items for all start-up activities, and change state to ``active''.
\end{description}
If you comment on an action you take, the comment will appear in the event
log.

The \emph{work item list} contains all work items ever started in this workflow
instance. For each, it provides links to the work item page and the
corresponding activity. It also displays each work item's state, the time it
was started and, for completed items, the time it finished.

The \emph{event log} reports on the instance's creation and the actions that were
performed on it later, such as starting or restarting it. This log does not
contain information on any individual work items.

\subsection{Controlling a work item}

The work item management view is similar to that for workflow instances. It
starts out with information on the work item: its state, activity, and
instance, the users relevant to this work item, the work item that created it,
and all work items created by it.

Other than workflow instances, work items offer two kinds of actions to
perform: user actions and management actions. User actions depend on the type
of work item, e.g. ``complete'' for a task; automatic work items do not have
any user actions. You cannot comment on a user action you perform from the
ZMI; this is only possible from the Plone interface. User actions can only be
taken for active work items, they are merely listed for other states.

The following management actions are available for work items:
\begin{description}
\item[fall out] Mark the work item as being in an exceptional state. This also
  makes the workflow instance fall out.
\item[terminate] Terminate the work item without waiting for it to complete
  successfully.
\item[restart] Restart the work item all over. This puts it back into the
  state ``active''. It also creates work items for its start activities again.
\end{description}
For these actions, there is an input field for comments.

The work item event log lists the actions, including comments, that have been
performed on the work item. This is the same information that can be found in
the detailed work item overview provided by the Plone user interface.

\section{Administration Tools}

The "`tools"' tab on the \member{workflow\_manager} provides utilities that
allow you to smoothly administer your workflows. They are especially helpful
on large installations.

\begin{memberdesc}{Clean terminated and stale instances} Removes workflow
    instances that are no longer of interest. Terminated instances are those
    cast away by a manager, stale instances those whose content object no
    longer exists or instances where the process definition does not exist. 
\end{memberdesc}

\begin{memberdesc}{Run sanity check}
Runs sanity checks on all AlphaFlow-aware content in the
portal. It reports missing or duplicate references between content objects and
workflow instances.
\end{memberdesc}

\begin{memberdesc}{Bulk drop-in}

In the case of many processes falling out, the bulk drop-in allows you to
\emph{try} dropping in all instances that are fallen out at the moment. It will
ignore all instances that still have work items fallen out. Use this button
after reparing many workflows that were fallen out.

\end{memberdesc}

\begin{memberdesc}{Restart helper}

In the case of a workflow or application failure resulting in many work items
falling out you can use the restart helper to reactive all work items that
belong to a certain activity of a certain process and are fallen out. The work
items are restarted. Use the \member{bulk drop-in} feature to let those
workflows continue.

\end{memberdesc}
