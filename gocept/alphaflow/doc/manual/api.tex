\chapter{API reference}

  \section{QuickInstaller support}
    
    When writing your own products, the AlphaFlow installation code provides
    a convenience method to automatically import your ALF workflow definitions.

    The method is defined in the module \module{Products.AlphaFlow.Extensions.Install}
    and can be imported from there.

    \begin{funcdesc}{install_process_definitions}{out, globals}
      The method looks for a \verb|workflows| directory in your product 
      and imports all process definitions with the file extension \bfcode{alf}.
      \var{out} is a file-like object containing the installation protocol. \var{globals}
      are the globals of your product that point to the product's directory.
    \end{funcdesc}

  \section{\module{IProcessManager} --- Manages workflow definitions and instances.}

  \declaremodule{standard}{ProcessManager}
  \modulesynopsis{Workflow definition and process instance management}
 
  \begin{classdesc}{IProcessManager}{}

    \begin{memberdesc}{processes}
        Folder-like object containing workflow definitions.
    \end{memberdesc}

    \begin{memberdesc}{instances}
        Folder-like object containing workflow instances.
    \end{memberdesc}

    \begin{memberdesc}{schema}
        Path to RelaxNG schema for ALF validation.
    \end{memberdesc}

    \begin{funcdesc}{initProcess}{process_id, object}
        Returns newly created process instances from the definition
        \var{process_id} and associates it with the given object.
    \end{funcdesc}

    \begin{funcdesc}{addProcess}{process_id}
        Creates a new process definition given by \var{process_id}.
    \end{funcdesc}

    \begin{funcdesc}{deleteProcess}{process_id}
        Removes the process definition given by \var{process_id}.
    \end{funcdesc}

    \begin{funcdesc}{getProcess}{process_id, default}
        Return a process definition given by \var{process_id}. 
        \var{Default} will be returned if no process with 
        \var{process_id} can be found. The method raises an 
        AttributeError, if no \var{default} parameter 
        is given.
    \end{funcdesc}

    \begin{funcdesc}{listProcesses}{}
        Return a sequence of all process definitions.
    \end{funcdesc}

    \begin{funcdesc}{renderProcess}{process_id, format="gif", kind=None, REQUEST=None}
        Render a process definition given by \var{process_id} as a 
        graph. The \var{format} parameter can be one of \var{svg, gif, 
        None}. The \var{kind} parameter can be of \var{detailed}, 
        \var{minimal} and \var{None}. 
    \end{funcdesc}

    \begin{funcdesc}{restrictedListProcesses}{context}
        Return a sequence of process definitions which may be instantiated in
        the given context.
    \end{funcdesc}

    \begin{funcdesc}{getWorkItems}{user}
        Return all work items that are relevant to the given user.
    \end{funcdesc}

    \begin{funcdesc}{getWorkItemsForCurrentUser}{}
        Return all work items that are relevant to the current user.
    \end{funcdesc}

    \begin{funcdesc}{importWorkflowFromXML}{process_id, xmlfile}
        Imports a process definition from an xml file. \var{Xmlfile} 
        must be a file or fileupload instance. The method returns the 
        imported process definition instance.
    \end{funcdesc}
    
    \begin{funcdesc}{renderInstance}{instance_id, format="gif", kind=None, REQUEST=None}
        Render a process instance given by \var{instance_id} as a 
        graph. The \var{format} parameter can be one of \var{svg, gif, 
        None}. The \var{kind} parameter can be of \var{detailed}, 
        \var{minimal} and \var{None}. If the value is \var{None}, it 
        defaults to \var{config.GRAPH_DISPLAY}.
    \end{funcdesc}

    \begin{funcdesc}{pingCronItems}{}
        Trigger all due \class{Alarm} work items.
    \end{funcdesc}

    \begin{funcdesc}{sanityCheck}{}
        Ensures that everything is in a sane state.
    \end{funcdesc}
    
  \end{classdesc}

  \section{\module{IProcess} --- A process definition.}
  \declaremodule{standard}{Process}
  \modulesynopsis{A process definition.}

  \begin{classdesc}{IProcess}{}
    
    \begin{funcdesc}{listActivityIds}{}
      Returns a list of all activity ids used in the process definition.
    \end{funcdesc}

    \begin{funcdesc}{addActivity}{id, activity_type}
      Adds a new activity to this process definition or raises a 
      KeyError if the \var{activity_type} is unknown. The 
      \var{activity_type} can be one of the listed activities in 
      \ref{sec:activityref}. The \var{id} parameter ist mandatory.
    \end{funcdesc}

    \begin{funcdesc}{acquireProcess}{}
      Returns the process instance from the acquisition chain.
    \end{funcdesc}

  \end{classdesc}

  \section{\module{IContentObjectRetriever} --- Get the content object in various representations}

    \declaremodule{standard}{IContentObjectRetrieverBase}
    \modulesynopsis{Get the content object in various representations}

    \begin{classdesc}{IContentObjectRetrieverBase}{}

      \begin{funcdesc}{getContentObject}{}
        Return the associated content object or None.
      \end{funcdesc}

      \begin{funcdesc}{getContentObjectUID}{}
            Return the associated content object's UID.
      \end{funcdesc}

      \begin{funcdesc}{getUrl}{}
            Return the URL of the associated content object.
      \end{funcdesc}

      \begin{funcdesc}{getContentObjectPath}{}
            Return the path of the associated content object.
      \end{funcdesc}

      \begin{funcdesc}{getContentObjectUIDBrain}{}
          Return brain from \var{uid_catalog} for associated content object.
      \end{funcdesc}

      \begin{funcdesc}{getContentObjectPortalCatalogBrain}{}
          Return the \var{portal_catalog} brain for the content object.
      \end{funcdesc}

    \end{classdesc}

  \section{\module{IInstance} --- A workflow instance}
    \declaremodule{standard}{IInstance}
    \modulesynopsis{A process instance.}

    \begin{classdesc}{IInstance}{id, object, process_id}
     
      \begin{funcdesc}{start}{comment}
        Start the process instance, which is done automatically after 
        the instance has been added.
      \end{funcdesc}

      \begin{funcdesc}{reset}{comment}
        Terminate all work items and reset the instance. 
      \end{funcdesc}

      \begin{funcdesc}{terminate}{comment}
        Terminate all work items and leave the instance as terminated.
      \end{funcdesc}

      \begin{funcdesc}{restart}
        Reset the process instance and start it again.
      \end{funcdesc}

      \begin{funcdesc}{dropin}
        Go back to the active state when in failed state. In order to 
        drop back in, there must be no work items in failed state.
      \end{funcdesc}

      \begin{funcdesc}{getWorkItem}{id}
        Return the work item with the given \var{id}. This method 
        raises an AttributeError if the work item doesn't exist. 
        Furthermore, an Unauthorized exception is raised, if the 
        current user does not have the \constant{WORK_WITH_PROCESS} 
        permission on the requested work item.
      \end{funcdesc}

      \begin{funcdesc}{unrestrictedGetWorkItem}{id}
        Return the work item with the given \var{id}.
      \end{funcdesc}

      \begin{funcdesc}{getWorkItems}{state="active", activity_id=None}
        Return a list of all work items in the given \var{state}. If 
        \var{activity_id} is not None, all work items that belong to 
        the activity with the given \var{id} are returned. Possible 
          work item states are described at \ref{workitemstates}.
      \end{funcdesc}

      \begin{funcdesc}{getWorkItemIds}{state="active"}
        Return a list of the ids of all work items in the given \var{state}.
        Possible work item states are described at \ref{workitemstates}.
      \end{funcdesc}

      \begin{funcdesc}{createWorkItems}{activity_ids, source, content_object=None}
        Create new work items for the activities with the given 
        \var{activity_ids}. This method raises a KeyError, if no 
        activity with this id is known. It returns a tuple containing 
        the ids of the created work items.
      \end{funcdesc}

      \begin{funcdesc}{changeState}{state, comment}
        Changes the \var{state} of the work item. This is recorded to the event
        log by using the given \var{comment}.
        Possible work item states are described at \ref{workitemstates}.
      \end{funcdesc}

      \begin{funcdesc}{recordAction}{action, comment}
        Record an \var{action} and a \var{comment} to the event log.
      \end{funcdesc}

      \begin{funcdesc}{notifyWorkItemStateChange}{workitem}
        Notify the process instance that the \var{workitem} has changed 
        its state.
      \end{funcdesc}

      \begin{funcdesc}{getProcess}{}
        Return the process definition this is an instance of.
      \end{funcdesc}

      \begin{funcdesc}{getInstance}{}
        Returns \var{self}.
      \end{funcdesc}

      \begin{funcdesc}{updateWorkitemsAndContentObjects}{}
        Update everything affected after editing the workflow configuration.
      \end{funcdesc}

    \end{classdesc}

  \section{\module{IAlphaFlowed} --- A content object managable by AlphaFlow}

    \declaremodule{standard}{IAlphaFlowed}
    \modulesynopsis{A content object managable by AlphaFlow}

    \begin{classdesc}{IAlphaFlowed}{}
        
      \begin{funcdesc}{getSuitableProcesses}{}
        Return a list of suitable process definitions.
      \end{funcdesc}

      \begin{funcdesc}{hasInstanceAssigned}{}
        Return whether a process is already assigned to this object or not.
      \end{funcdesc}

      \begin{funcdesc}{assignProcess}{process_id}
        Assign a new instance of the process definition with the given 
        \var{process_id} to this object.\note{This method does nothing, 
        if the object already has a process instance assigned.}
      \end{funcdesc}

      \begin{funcdesc}{getInstance}{}
        Return the currently assigned process instance. This method 
        raises a KeyError of no process instance is currently assigned.
      \end{funcdesc}

      \begin{funcdesc}{getAllInstances}{}
        Return a list of all assigned process instances, both completed 
        and running.
      \end{funcdesc}

      \begin{funcdesc}{getWorkItemsForCurrentUser}{}
        Returns a list of work items for this object.
      \end{funcdesc}

      \begin{funcdesc}{getWorkItem}{id}
        Return a work item with the given \var{id} from the currently 
        to this content object attached process instance. This method 
        can raise three different errors:
        \begin{description}
          \item[AttributeError] if there is no work item with the given \var{id}
          \item{ValueError} if no active process instance is attached
          \item{Unauthorized} if the current user does not have the 
          \constant{WORK_WITH_PROCESS} permission on the requested work item
        \end{description}
      \end{funcdesc}

      \begin{funcdesc}{alf_clearInstances}{}
        Detach all (current and old) instances.
      \end{funcdesc}
      
    \end{classdesc}
    
  \section{\module{IActivity} --- A workflow activity}

    \declaremodule{standard}{IActivity}
    \modulesynopsis{An entity in a process definition.}
     
    \begin{classdesc}{IActivity}{}
       
      \begin{funcdesc}{getPossibleChildren}{}
        Returns a list of all ids of activities that may be 
        instantiated as successors by instances of this activity.
      \end{funcdesc}

      \begin{funcdesc}{getConfigurationSchema}{content}
        Getter method to retrieve the instance configuration schema. This also 
        allows for different activities to have programmatic influence 
        over how instance configuration schemas look like.
      \end{funcdesc}

      \begin{funcdesc}{graphGetPossibleChildren}{}
        Returns a list of possible successor activities as dictionaries. Used
        for graph generation.
      \end{funcdesc}

      \begin{funcdesc}{graphGetStartActivities}{}
        Returns a list of activity titles which are 
        \var{startActivities} of this one.
      \end{funcdesc}

      \begin{funcdesc}{acquireActivity}{}
        Returns the activity instance from the acquisition chain.
      \end{funcdesc}

      \begin{funcdesc}{generateWorkItems}{source, content\_object}
        Instanciates work items for an activity. 

        Returns a list of work items.

        This method works as a factory to create work items for an activity
        when ``createWorkItems'' is called.

        The \class{BaseActivity} has a basic method that creates a single work
        item that is identified by the \var{activity_type}. In more complex situations 
        this method allows to create work items of any type and in any amount
        while only a single activity is configured.

        An example of a custom \method{generateWorkItems} method can be found
        in \file{Products/AlphaFlow/activities/simpledecision.py}.
      \end{funcdesc}
    \end{classdesc}


  \section{\module{IAssignableActivity} --- An assignable workflow activity}

    \declaremodule{standard}{IAssignableActivity}
    \modulesynopsis{An activity which holds information for user assignment.} %TODO urks
     
    \begin{classdesc}{IAssignableActivity}{}
        \begin{funcdesc}{getPossibleAssignees}{}
            Returns a list of possible assigned users. \note{This 
            method works only for roles.}
        \end{funcdesc}
    \end{classdesc}
    
  \section{\module{IWorkItem} --- A work item}
    
    \declaremodule{standard}{IWorkItem}
    \modulesynopsis{The instance of an activity.}

    \begin{classdesc}{IWorkItem}{id, activity_id, content_object=None}

      \begin{funcdesc}{getGeneratedWorkItems}{}
          Returns a list of work items this work item has generated.
      \end{funcdesc}

      \begin{funcdesc}{getActions}{}
          Returns a list of actions the user may perform on this work 
          item. This method can be used for the Plone action menu.
      \end{funcdesc}

      \begin{funcdesc}{getActionById}{id}
          Return the action with the given id.
          Raises KeyError if an action with the given id does not exist.
      \end{funcdesc}

      \begin{funcdesc}{isRelevant}{user}
          Return whether this work item is relevant for the given user.
      \end{funcdesc}

      
      \begin{funcdesc}{listRelevantUsers}{}
          Return a list of relevant user ids.
      \end{funcdesc}

      \begin{funcdesc}{isChildOf}{workitem_id, workitem}
        Returns \var{True} if this work item was generated by the given work item.
        \note{The method accepts only workitem_id or workitem.}
      \end{funcdesc}
      
      \begin{funcdesc}{changeState}{state, comment, exc_info=None}
          Changes the state of the work item to: \var{state}. Possible 
          work item states are described at \ref{workitemstates}.
      \end{funcdesc}

      \begin{funcdesc}{recordAction}{action, comment}
          Record an \var{action} and a \var{comment} to the event log.
      \end{funcdesc}
      
      \begin{funcdesc}{getActivity}{}
          Returns the corresponding activity to this work item.
      \end{funcdesc}

      \begin{funcdesc}{getActivityTitleOrId}{}
          Return the \var{title} or \var{id} of the activity this is an 
          instance of.
      \end{funcdesc}

      \begin{funcdesc}{getActivityConfiguration}{}
          Return the configuration for this activity in the context of 
          this process instance.
      \end{funcdesc}

      \begin{funcdesc}{getParent}{}
          Returns the parent work item or None if this is a root windows.
      \end{funcdesc}

      \begin{funcdesc}{getShortInfo}{}
          Returns a short information text. %TODO example?
      \end{funcdesc}

      \begin{funcdesc}{getStatusInfo}{}
          Returns a short status information text.
      \end{funcdesc}

      \begin{funcdesc}{onStart}{}
          Trigger that gets called after the work item is started.
      \end{funcdesc}

      \begin{funcdesc}{onTerminate}{}
          Trigger that gets called after the work item is terminated.
      \end{funcdesc}

      \begin{funcdesc}{onFailure}{}
          Trigger that gets called after the work item fell out.
      \end{funcdesc}

      \begin{funcdesc}{beforeCreationItems}{items, parent}
          Trigger that gets called before new work items get active. 
          Other work items can veto on the creation of those items and 
          return a list of ids as a veto. After all work items have 
          been triggered, the vetoed work items get removed again and 
          never become active.
      \end{funcdesc}

      \begin{funcdesc}{getActionVocabulary}{}
          Returns a list of tuples of IAction objects.
      \end{funcdesc}

      \begin{funcdesc}{notifyWorkItemStateChange}{workitem}
          Receives a notification that the work item \var{workitem} has 
          changed it's state.
      \end{funcdesc}

      \begin{funcdesc}{notifyAssigneesChange}{}
          Notifies the work item that the assignees might have change.
      \end{funcdesc}

    \end{classdesc}
  
  \section{\module{IAssignableWorkItem} --- A base assignable work item}
    
    \declaremodule{standard}{IAssignableWorkItem}
    \modulesynopsis{Work items which are assignable to users.}

    \begin{classdesc}{IAssignableWorkItem}{}

        \begin{funcdesc}{getGroupedSchema}{}
            Returns a sequence of \class{IFieldGroup} instances. The 
            method aggregates configuration schemas from all activities 
            which are configured by this work item (incl. it's own 
            schema) and returns a scheme, which is grouped by activity.
        \end{funcdesc}

        \begin{funcdesc}{getViewUrl}{}
            Returns an URL to view the page which lets the user handle 
            the work item.
        \end{funcdesc}

        \begin{funcdesc}{listMemberWithRolesOnContentObject}{roles}
            Returns a list of member ids, who have one of the given \var{roles}.
        \end{funcdesc}

    \end{classdesc}

  \section{\module{IAutomaticWorkItem} --- An automatic work item}
    
    \declaremodule{standard}{IAutomaticWorkItem}
    \modulesynopsis{Work items for automatic activities.}

    \begin{classdesc}{IAutomaticWorkItem}{}
        
        \begin{funcdesc}{run}{}
            Performs the actual automatic activity.
        \end{funcdesc}
    
    \end{classdesc}

\section{\module{IAction} --- A work item action}

    \begin{classdesc}{IAction}{Interface}

        \begin{memberdesc}{id}
            Id of the action. Must be unique for each work item.
        \end{memberdesc}

        \begin{memberdesc}{title}
            Title of the action. May be unicode.
        \end{memberdesc}

        \begin{memberdesc}{url}
            URL to trigger the action.
        \end{memberdesc}

        \begin{funcdesc}{__call__}{}
            Execute the action.

            Semantics are the same as in calling the url.
        \end{funcdesc}
    \end{classdesc}
