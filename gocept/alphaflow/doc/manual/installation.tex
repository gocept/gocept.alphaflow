\chapter{Installation}

AlphaFlow is not hard to install, but if you need help, check the chapter
``Getting support'' on how to get someone to help you.

\section{Prerequisites}

AlphaFlow depends on a couple of software products:

\begin{itemize}
    \item Zope (required: 2.9.6+) -- \url{http://www.zope.org}
    \item Plone (required: 2.5.2+) -- \url{http://www.plone.org}
    \item optional: pygraphviz for visual representations in the workflow
        editor -- \url{http://www.graphviz.org}
    \item optional: PloneTestCase for running the unit tests
\end{itemize}

To develop applications with AlphaFlow you should be familiar with:

\begin{itemize}
    \item the Python programming language
    \item filesystem-based product development for Zope 2, CMF and Plone using Archetypes
    \item a basic understanding of workflow in general
\end{itemize}

\section{Installing}

\begin{enumerate}
  \item Create a new Zope instance.
  \item Put Plone into the \file{Products/} directory.
  \item Unpack the AlphaFlow archive into the \file{Products/}
    directory.
  \item Start the Zope server.
  \item Create a new Plone site.
  \item Install AlphaFlow and (optional: the Procurement) product via QuickInstaller.
\end{enumerate}

After successfully installing AlphaFlow you will find three new objects in
your Plone site: \samp{workflow\_manager}, \samp{workflow\_editor}, and
\samp{workflow\_catalog}. Additionally, AlphaFlow has added a skin layer that
customizes Plone's user interface for AlphaFlow.

\subsection{Installing support for time based activities}

If you want to use time based triggers, like the alarm activity,  you have to
set up an external trigger mechanism, typically a combination of cron and wget.
An example crontab entry to check for triggers every hour looks like this:

\begin{verbatim}
0  *  * * *     wget -S -O - http://<zopeserver>:<zopeport>/<plonesite>/workflow_manager/pingCronItems
\end{verbatim}

You might want to call the trigger more often, based on the time resolution your application needs.

\subsection{AlphaFlow support for the standard content types}
\label{sec:install-patch}

AlphaFlow can only be used with content types that have special properties
which make them AlphaFlow-aware. From Plone 2.1 final on, Plone's default
content types are patched accordingly upon AlphaFlow installation.

To avoid patching the default content types, edit the file
\file{customconfig.py} in the \file{Products/AlphaFlow} directory to read:

\begin{verbatim}
# Only for Plone 2.1: Do you want to patch Plone's default content types to
# use AlphaFlow?
PATCH_PLONE_TYPES = False
\end{verbatim}

To use AlphaFlow with Plone prior to version 2.1 final, you need some
additional Plone product that integrates AlphaFlow into your portal. One such
product is the demo application included in the AlphaFlow distribution.

\subsection{Installing the demo application}

If you want to use the demo application, you also should also:

\begin{enumerate}
  \item Copy or link the Procurement product found in AlphaFlow's
    \file{doc/example/Procurement/} directory into the \file{Products/}
    directory.
  \item Restart the Zope server.
  \item Install the Procurement product via QuickInstaller.
\end{enumerate}


\section{Testing}

AlphaFlow comes with a set of unit tests that you might want to run, to see if your 
environment works with AlphaFlow. In your Zope instance directory run:

\begin{verbatim}
bin/zopectl test --dir=Products/AlphaFlow/
\end{verbatim}

This will start the unit test runner and might take a couple of minutes. It will report you
if everything was OK, or if an error occured. As AlphaFlow supports different environments
you might meet several \code{DeprecationWarning} messages. This is ok, as long as the overall output looks like this:

\begin{verbatim}
----------------------------------------------------------------------
Ran 91 tests in 36.945s

OK
\end{verbatim}
