\chapter{Activity Reference}\label{sec:activityref}

As AlphaFlow processes are assembled from a variety of specialised activities,
every activity has its own uses and configuration variants. This section
describes the existing standard activities.


\section{Structure of the activity reference}

In the following sections, we describe every activity in a common format:

\begin{description}
    \item[Title and purpose] of each section will show the name of the activity and a very short 
        description of its purpose.
    \item[Introduction] will introduce the activity and its use cases.
    \item[Activity configuration] describes all options that can be specified for an activity, e.g. in ALF.
    \item[Instance configuration] describes options that can be specified on a per-workflow-instance level.
    \item[Work item actions] explains the possible interaction with users, external systems or programmers while a work item of an activity is active. 
    \item[ALF example] An example is given how a configuration of this activity in an ALF file would look like.
\end{description}

\section{XML workflow definitions}

The preferred way of defining workflows for AlphaFlow is by creating XML files
in the ALF format and importing them. This section is a short introduction
about the basics of the ALF format and is supplemented by the ``ALF example''
sections in this reference.

As ALF is an XML format, you should stick to the official XML specifications.

An ALF file always has a \member{workflow} element as the root element.  All
activity definitions are child elements of the \member{workflow} element.

The \member{workflow} element has following attributes:

\begin{memberdesc}{id}
  A suggestion for the Zope id of the workflow definition. It may be
  overridden on import.
\end{memberdesc}

\begin{memberdesc}{title}
A title to be used in the user interface.
\end{memberdesc}

\begin{memberdesc}{description}
A description to be used in the user interface.
\end{memberdesc}

\begin{memberdesc}{startActivity}
A list of activity IDs. Work items for these activities
  are started when an instance of the workflow is started.
\end{memberdesc}

\begin{memberdesc}{onlyAllowRoles}
Only members with one of these roles may start instances
  of this workflow. If omitted, anyone may do so.
\end{memberdesc}

\begin{memberdesc}{object_name}
   An optional name for the content object that is used in any
  TALES expressions that may appear in the workflow definition. It defaults to
  ``object''.
\end{memberdesc}

\begin{memberdesc}{commentfield}
 Controls the view of the commentfield. Set it to ``required'' to make 
 the user input required. Set it to ``hidden'' to hide the commentfield 
 from the UI.
\end{memberdesc}

ALF example:

\begin{verbatim}
<?xml version="1.0" encoding="utf-8"?>
<workflow id="simple_review"
          title="Simple review"
          description="A simple review workflow."
          startActivity="switch_configuration"
          onlyAllowRoles="Owner"
          object_name="document">
    ...
</workflow>
\end{verbatim}

\section{Common configuration options}

AlphaFlow provides common functionality for all activities and therefor
includes common configuration options for all activities.

\begin{memberdesc}{id}
    The id is the unique reference for an activity within a workflow definition. It is used by
    other configuration options to refer to a certain activity. The id must conform to the XML attribute ``id'' and
    the rules for Zope object ids.
\end{memberdesc}

\begin{memberdesc}{title}
    The title is a title that is displayed to the user. It will appear in the work list, the workflow menu
    and other places visible to the user. The title can be any unicode string.
\end{memberdesc}

\begin{memberdesc}{sort} This priority, given as an integer, is used when
    displaying configuration activities. It sorts the configuration schemas of
    all included activities by their priority. E.g. first display ``select
    author'', then ``select reviewer'' then ``decide whether recursion is
    applied''.
\end{memberdesc}

\begin{memberdesc}{view\_url\_expr} A TALES expression that returns a URL the
    user is redirected to when clicking on the work item in the work item list.
    If not specified, the user will be redirected to the content object.
\end{memberdesc}

\begin{memberdesc}{startActivity}
    When a work item for this activity is created, all activities listed as ``startActivity'' will be started.
    Then the created work item is started. This allows for useful grouping of actions and to refer to common
    entry points without having to know its prerequisites. E.g. you can refer to ``write a document'' without knowing
    that ``set permissions'' and ``set dcworkflow state'' should be run as well. This also makes refactoring 
    of a workflow easier.
\end{memberdesc}

ALF example:

\begin{verbatim}
<task id="write_document"
    title="Write a document"
    sort="1"
    viewUrlExpression="string:${object/absolute_url}/explain_writing_a_document"
    startActivity="dc_private perm_authoring"
    completion_activity="review">
    <assignees kind="possible" roles="Member"/>
</task>
\end{verbatim}

\section{Assignable Activities}

  Assignable activities are, as the name says, assigned to a user or external
  system to perform a real world task, outside of AlphaFlow. Those activities
  typically provide a user interface and/or API to interact with.

  \subsection{Common configuration options for assignable activities}

  All assignable activities have common configuration options to control
  user and role assignments within a workflow.
    
  \begin{memberdesc}{assignees}

        Determines the responsible users for an activity. The assignee
        configuration supports selecting users based on role or an expression
        determining individual user ids. Additionally a certain set of users
        may be selected for a certain workflow instance from all users with a
        role.

        In ALF files the assignee configuration is given as a sub-element for
        an activity and may only be given once.

        \begin{description}

            \item[kind='actual', roles='\ldots'] All users with a given role
                will be responsible for the work item. Typically only one user
                will actually perform the activity. But special activities,
                like \code{Decision} may use the list of all assigned users to
                deduce more complex behaviour. 

            \item[kind='actual', expression='\ldots'] All users returned by the
                given TAL expression will be responsible. The TAL expression
                must return a list of user ids. The TAL expression is evaluated
                when the work item is started and cached afterwards. It should
                always return the same list of users, otherwise the systems
                behaviour is currently not guaranteed to be predictable.

            \item[kind='possible', roles='\ldots']

                The actual list of users will be selected when running the
                workflow. A list of all users having at least one of the
                specified roles will be given and the user configuring the
                running workflow can select one ore more users to be assigned
                to the work item.
                
        \end{description}

        Examples:

        \begin{enumerate}
            \item All users with the ``Finance'' role are assigned:
\begin{verbatim}
<assignees kind="actual" roles="Finance"/>
\end{verbatim}
            \item The users ``Peter'', ``Paul'' and ``Mary'' are assigned:
\begin{verbatim}
<assignees kind="actual" expression="python:['Peter', 'Paul', 'Mary']"/>
\end{verbatim}
            \item One or more users with the role ``Reviewer'' may be selected when running the workflow:
\begin{verbatim}
<assignees kind="possible" roles="Reviewer"/>
\end{verbatim}
        \end{enumerate}
  \end{memberdesc}

  \begin{memberdesc}{contentRoles}

        Assigns one or more local roles to every user assigned to the work item.

        The local roles are given on the content object of the work item and
        valid as long as the work item is active.

        In ALF files the contentRoles configuration is given as an attribute
        for an activity.

        Examples:

        The reviewers for ``Marketing'' will get the role ``MarketingReviewer''
        while the task ``review\_marketing'' is active.

        \begin{enumerate}
            \item The assigned user will get the ``MarketingReviewer'' role while assigned to the ``review\_marketing'' task:
\begin{verbatim}
<task id="review_marketing" contentRoles="MarketingReviewer"/>
\end{verbatim}
        \end{enumerate}

  \end{memberdesc}

  \subsection{\module{Task} --- Perform a task}

  \declaremodule{standard}{task}
  \modulesynopsis{Assign tasks to users}

  The \module{Task} is the most basic activity a user will be confronted with.
  It allows to ask the assigned users to carry out some real world task. This
  can range from editing a document in Plone, ordering new paper for your
  office, or sending a letter to a customer.

  The assigned users will see a work item in their work list telling them what
  to do. When the task is completed, one of them selects
  ``\guilabel{Complete}'' from the workflow menu which will complete the work
  item and start the \code{completion\_activity}.

  \subsubsection{Activity configuration}

            
    \begin{memberdesc}{title} The title is used as a static description of the
            task. This will be displayed in the work list for the assigned
            users together with the title of the content object.
    \end{memberdesc}
          
    \begin{memberdesc}{completion\_activity} Lists the activities to be started
        after the task has been completed.
    \end{memberdesc}

  \subsubsection{Instance configuration}

    \begin{description}
            
        \item[task] If the task (e.g. ``Write a document'') is configured when
            running a workflow, it can be further specified and annotated for
            the assigned user (e.g. ``Write 400 words about dead parrots.'') 

    \end{description}

  \subsubsection{Work item actions}

    \begin{funcdesc}{complete}{comment}
        
      \function{Complete} is the only action \module{Task} supports. It is the
      signal that the user has performed the required activity. Additionally,
      the user can provide a comment when completing a task.

    \end{funcdesc}

  \subsubsection{ALF example}

\begin{verbatim}
<task id="write_document"
    title="Write document"
    completion_activity="review">
    <assignees kind="actual" expression="python:[object.Creator()]" />
</task>
\end{verbatim}

  \subsection{\module{NTask} --- A task with multiple completion options}

  \declaremodule{standard}{ntask}
  \modulesynopsis{A task that lets the user choose how the process continues.}

  The \module{NTask} activity is an extension to the \module{Task} that allows
  the user to choose how the workflow should continue. It also allows to
  provide your own labels for the workflow menu that the user sees.

  The different options to continue the workflow are called ``exits''.

  Examples:

  \begin{itemize}
      
      \item Ask someone to ship a package. He can select ``successfully
          shipped'' or ``shipping didn't happen'' to signal whether he
          successfully shipped the package or not.

      \item Make someone decide whether a customer request email has to be handled
          in a certain way or other.

  \end{itemize}

  \subsubsection{Activity configuration}

    \begin{description}
        \item[exit] A sub-element with an \var{id}, \var{title} and a list of \var{activities}. Can be given multiple times
            to supply multiple exits.
    \end{description}

  \subsubsection{Instance configuration}

    \begin{description}
        \item[task] If the task (e.g. ``Write a document'') is configured when
            running a workflow, it can be further specified and annotated for
            the assigned user (e.g. ``Write 400 words about dead parrots.'') 

    \end{description}

  \subsubsection{Work item actions} %user interface

    Work item actions are determined by the given \var{exits}.
    
  \subsubsection{ALF example}

\begin{verbatim}
    <ntask id="edit" title="Modify document">
        <exit id="make_public" title="Make public" activities="public"/>
        <exit id="make_pending" title="Approve Document" activities="pending"/>
        <exit id="make_private" title="Close Document" activities="private"/>
        <assignees kind="actual" roles="Member" />
    </ntask>
\end{verbatim}

 \subsection{\module{Configuration} --- Configure other activities}

 \declaremodule{standard}{configuration}
 \modulesynopsis{Configures other activities}
 
  The \module{Configuration} activity provides some \emph{meta} functionality:
  it allows a workflow user to configure parts of the workflow while running
  it.

  Useful examples include: 
  
  \begin{itemize}
      \item selecting specific users for a certain review or decision
      \item selecting if a folder should be published recursively or not
      \item specifying a task more precisely
  \end{itemize}

  The \module{Configuration} activity is responsible to configure the
  parameters given in the ``Instance configuration'' sections of other
  activities.
  
  \subsubsection{Activity configuration}


    \begin{memberdesc}{configures}
        A list of activities that have to be configured by this
          configuration activity.
    \end{memberdesc}
          
    \begin{memberdesc}{configures\_all} A marker that all activities that have
        instance configuration options should be configured by this activity.
        Useful at the beginning of a workflow.
    \end{memberdesc}

    \begin{memberdesc}{continue\_activity} A list of activities that shall be
        started after the configuration.  
    \end{memberdesc}

  \subsubsection{Instance configuration}

  Not applicable.
    
  \subsubsection{Work item actions} %user interface


    \begin{funcdesc}{configure}{comment}
        
      \function{configure} is the only action \module{Configuration} supports.
      It assumes that the configuration data was previously saved using the
      standard forms generated by AlphaFlow.

    \end{funcdesc}

  \subsubsection{ALF Example}

  In this example, the \module{Configuration} asks the owner of an object
  (likely a document) to select another member who should write a document and
  starts the corresponding task afterwards.

\begin{verbatim}
<task id="write_doc"
    title="Write document">
    <assignees kind="possible" roles="Member"/>
</task>

<configuration 
    id="assign_task" 
    title="Select a user to write the document"
    configures="write_doc"
    continue_activity="write_doc">
  <assignees kind="actual" roles="Owner" />
</configuration>
\end{verbatim}

 \subsection{\module{Decision} --- Let a group of people make a decision}
 
 \declaremodule{standard}{decision}
 \modulesynopsis{Let a group of people make a decision}
 
 A \module{Decision} allows to let a a group of people decide for or against
 some proposed action. The the possible exits are named ``accept'' and
 ``reject'' accordingly. A \module{Decision} works much like casting a vote.
 More policies could extend the decision to allow majority votes or similar
 schemes.
 
 Two different policies for making a decision are implemented:

 \begin{description}

     \item[first\_yes:] As soon as someone agrees, the whole decision will be
         agreed on.
         
     \item[all\_yes:] Everybody has to agree to make the decision. As soon as a
         single user disagrees, the decision will be rejected.
         
 \end{description}

 In both schemes, the work item for a decision will be completed as soon as
 enough votes are casted so the result is known.
 
  \subsubsection{Activity configuration}

    \begin{memberdesc}{accept\_activity}
        Is a list of activities to be started when the decision is \emph{accepted}.
    \end{memberdesc}

    \begin{memberdesc}{reject\_activity}
      Is a list of activities to be started when the decision is \emph{rejected}.
    \end{memberdesc}

  \subsubsection{Instance configuration}

  No instance configurable options are available.
     
  \subsubsection{Work item actions} %user interface

    \begin{funcdesc}{accept}{comment}
        
        \function{Accept} Registers the currently logged-in user to have voted to \emph{accept}
            the proposed action.    

        \function{Reject} Registers the currently logged-in user to have voted to \emph{reject}
            the proposed action.    

    \end{funcdesc}

  \subsubsection{ALF Example}

\begin{verbatim}
<decision id="decide_witch"
  decision_notice="Decide whether she is a witch."
  accept_activity="burn_her"
  reject_activity="send_her_home"
  decision_modus="all_yes">
  <assignees kind="actual" roles="Peasants" />
</decision>
\end{verbatim}

%%%%%%%%%%%%%%%%%%%%%%%%%%%%%%%%%%%%%%%%%%%%%%%%%%%%%%%%%%%%%%%%%%%%%%%%%%%%%%%
\section{Automatic activities}
%%%%%%%%%%%%%%%%%%%%%%%%%%%%%%%%%%%%%%%%%%%%%%%%%%%%%%%%%%%%%%%%%%%%%%%%%%%%%%%
    
 Automatic activities carry out some procedure when the workflow activates
 them. They usually exist only for a short time period and continue the
 workflow immediately.
  
 \subsection{Common activity configuration}
    
 Some configuration parameters are shared amongst automatic activities through
 their base class. 

 \begin{memberdesc}{continue\_activity}
     This specifies the activities that should be started after the automatic work item
     finished.
 \end{memberdesc}

 \subsection{\module{DCWorkflow} --- Emulate DCWorkFlow-compatible ``states''}
 
 \declaremodule{standard}{dcworkflow}
 \modulesynopsis{An activity which emulates a DCWorkflow states.}
  
    To integrate AlphaFlow applications into Plone best this activity allows to
    set the DCWorkflow state for the content object. The content object should
    be connected with the \code{alphaflow\_fake} workflow to support this.

    The \module{DCWorkflow} activity does not use DCWorkFlows abilities to
    manage permissions or anything else. It only performs the necessary
    operations to display the state correctly in Plone.
 
    \subsubsection{Activity configuration}

     \begin{memberdesc}{status}
        Specifies the id of the workflow status to assume.
     \end{memberdesc}

  \subsubsection{Instance configuration}
 
  Not applicable.
     
  \subsubsection{ALF Example}

\begin{verbatim}
<dcworkflow id="private" status="private" />
\end{verbatim}
  
 \subsection{\module{Email} --- Notify users about events in the workflow}
 
 \declaremodule{standard}{email}
 \modulesynopsis{Notify users about events in the workflow}
  
 The \module{email} activity allows to inform users when an activity is
 started. It provides flexible ways to determine the recipients of the
 notification:

 \begin{itemize}
     \item notify the owner of the process
     \item notify the assignees of the next activities
     \item notify everybody with a certain role
 \end{itemize}

 The subject and the id of a template for the mail body can be given in the
 workflow definition as well.
 
 \subsubsection{Activity configuration}

    \begin{memberdesc}{template}
        An id of a template available from the Plone root object. Used to render the mail body. This
        is typically a DTML template.
    \end{memberdesc}

    \begin{memberdesc}{mailSubject}
        Specifies the subject printed in the email.
    \end{memberdesc}

    \begin{memberdesc}{recipient}
        Determines who is notified. Can be repeated multiple times. Following types are available:

        \begin{description}
                
            \item[next\_assignees] will notify all people who are responsible
                for the activities that \emph{follow} the notification. This
                allows to easily model something like: modify document, notify
                reviewers, review document \ldots

                Note that if the following activity is a route, it will use
                the contained activities to find out recipients.

            \item[previous\_assignees] will notify all people who were
              responsible for the email's parent activity.

                
            \item[owner] will only notify the actual owner of the content
              object. It willl \emph{not} notify the users with the Owner
              role.
                
            \item[actual\_role] will notify all people with at least one of 
                the given roles.

        \end{description}
    \end{memberdesc}

  \subsubsection{Instance configuration}

  Not applicable.
  
  \subsubsection{ALF example}

  Notify reviewers that they have to check something:

\begin{verbatim}
<email id="notify_reviewer" title="Notify about new review"
    template="default_email"
    mailSubject="Mail send by AlphaFlow">
    <recipient type="next_assignees"/>
</email>
\end{verbatim}

  Notify the process owner that the review was rejected: 

\begin{verbatim}
<email id="notify_owner" title="Review was rejected"
    template="reject_email"
    mailSubject="Your content was rejected">
    <recipient type="owner"/>
</email>
\end{verbatim}

  Notify all administrators that something bad happened:

\begin{verbatim}
<email id="notify_admin" title="Notify about the end of the world"
    template="trouble_email"
    mailSubject="Something bad happened">
    <recipient type="actual_role" roles="Manager"/>
</email>
\end{verbatim}

 \subsection{\module{Expression} --- Execute arbitrary code}
 \declaremodule{standard}{expression}
 \modulesynopsis{Execute arbitrary code}
 
 The \module{Expression} activity enables you to call any method
 your application needs within a workflow to talk to external systems,
 perform computation or other things. The Expression is given as a TALES expression. The calling context
 is initialized with those top level variables:

 \begin{tableii}{l|l}{constant}{Variable}{Description}
       \lineii{content, here, <object\_name>}{The content object the workflow instance manages.}
       \lineii{workitem}{The work item for the activity that executes the expression.}
       \lineii{portal}{The Plone root object.}
       \lineii{activity}{The expression activity.}
       \lineii{member}{The current member.}
       \lineii{alphaflow}{alphaflow/currentUser and alphaflow/systemUser.}
 \end{tableii}

  \subsubsection{Activity configuration}

    \begin{memberdesc}{expression}
        The TALES expression to execute when the activity is run.
    \end{memberdesc}

    \begin{memberdesc}{runAs}
        TALES expression returning a user or a username. The
        \texttt{expression} will be run under that user (like sudo)
    \end{memberdesc}

  \subsubsection{Instance configuration}

  Not applicable.

  \subsubsection{ALF example}

\begin{verbatim}
<expression id="annotate" 
    expression="python:object.addComment('Running expression: %s' % activity.title_or_id())"/>
\end{verbatim}
 
%%%%%%%%%%%%%%%%%%%%%%%%%%%%%%%%%%%%%%%%%%%%%%%%%%%%%%%%%%%%%%%%%%%%%%%%%%%%%
\subsection{\module{Permission} --- Control access to the content object}

\declaremodule{standard}{permission}
\modulesynopsis{Control access to the content object}
 
The \module{Permission} activity is able to change permission settings on the
content object. This can happen either by modifying the list of allowed roles
for a permission or giving a complete set of roles for a permission.
  
\subsubsection{Activity configuration}

\begin{funcdesc}{add}{permission, roles}
    Add the given permission for the given roles.
\end{funcdesc}

\begin{funcdesc}{remove}{permission, roles}
    Remove the given permission for the given roles.
\end{funcdesc}

\begin{funcdesc}{permission}{permission, roles, [acquire]}
    Set the given permission for exactly the given roles. Optionally, modify the 
    acquisition flag.
\end{funcdesc}

\subsubsection{Instance configuration}

Not applicable.

\subsubsection{ALF Example}

\begin{verbatim}
<permission-change id="perm_no_author">
    <remove name="Modify portal content" 
            roles="Owner"/>
    <add name="View"
         roles="Anonymous"/>
    <permission name="Access contents information" 
         roles="Anonymous Manager" 
         acquire="True"/>
</permission-change>
\end{verbatim}

%%%%%%%%%%%%%%%%%%%%%%%%%%%%%%%%%%%%%%%%%%%%%%%%%%%%%%%%%%%%%%%%%%%%%%%%%%%%%
\subsection{\module{Roleassign} --- Assign local roles to users on the content object}

\declaremodule{standard}{roleassign}
\modulesynopsis{Assign local roles to users on the content object}
 
The \module{Roleassign} activity is able to assign or remove given local roles for configurable 
users on the content object. 
  
\subsubsection{Activity configuration}

\begin{memberdesc}{roles}
    List of roles to assign users to on the content object. 
\end{memberdesc}

\subsubsection{Instance configuration}

    \begin{description}
        \item[users] Lets the user select which users the roles should be assigned.
    \end{description}

\subsubsection{ALF Example}

\begin{verbatim}
  <roleassign id="choose_manager"
              continue_activity="some_activity"
              roles="Manager"/>
\end{verbatim}

%%%%%%%%%%%%%%%%%%%%%%%%%%%%%%%%%%%%%%%%%%%%%%%%%%%%%%%%%%%%%%%%%%%%%%%%%%%%%%%
\subsection{\module{Switch} --- Programmatically select routes in the workflow}

\declaremodule{standard}{switch}
\modulesynopsis{Programmatically select routes in the workflow}
 
The \module{Switch} allows to programmatically decide to run alternative or
parallel routes in the workflow.  It is designed with the \code{switch}
statement from C/C++/Java in mind: You list multiple expressions that are
evaluated in the given order. If the mode is set to \var{all} the activities in
all given will be started for every condition that evalutes to \var{True}.  If
the mode is set to \var{first}, the activities for the first condition that
evaluates to \var{True} will be started and the evaluation of further
conditions will be stopped.
 
\subsubsection{Activity configuration}

\begin{memberdesc}{mode}
    oither set to \var{all} or \var{first}. 
\end{memberdesc}
\begin{memberdesc}{case}
    A TALES expression that evaluates to \var{True} or \var{False}.

    \notice{Check the \module{Expression} activity for the list of variables availabe in the expression context.}
\end{memberdesc}

\subsubsection{Instance configuration}
 
Not applicable.

\subsubsection{ALF example}

\begin{verbatim}
<switch id="check_money"
        mode="all">
  <case id="groceries"
        condition="python:account.current >= 20"
        activities="buy_groceries" />
  <case id="icecream"
        condition="python:account.current >= 30"
        activities="buy_icecream" />
  <case id="lotta_money"
        condition="python:account.current >= 500000"
        activities="buy_porsche" />
</switch>
\end{verbatim}

%%%%%%%%%%%%%%%%%%%%%%%%%%%%%%%%%%%%%%%%%%%%%%%%%%%%%%%%%%%%%%%%%%%%%%%%%%%%%%%
\subsection{\module{Alarm} --- Perform scheduled activities}
\declaremodule{standard}{alarm}
\modulesynopsis{Perform scheduled activities}
 
An \module{alarm} activity waits until a given deadline to run other
activities. The deadline is given as an expression and will be evaluated every
time when the alarm work item will be asked if the deadline was reached. You
can implement shifting deadlines by that.

An alarm is very suitable for:

\begin{itemize}
    \item reminders to re-review already published content
    \item create a milestone to escalate if someone doesn't react in a certain amount of time
\end{itemize}

\begin{notice} Please check the chapter ``Installation'' about installing support
 for time based activities.
\end{notice}
  
\subsubsection{Activity configuration}

\begin{memberdesc}{expression}
    Is a TALES expression that returns a \var{DateTime} object that specifies the
    deadline for this alarm.

    \note{Check the \module{Expression} activity for the list of variables availabe in the expression context.}
\end{memberdesc}

\subsubsection{Instance configuration}

Not applicable.

\subsubsection{ALF Example}

In this example, the \var{review} will be triggered 3 months after the object
was modified the last time:

\begin{verbatim}
<alarm id="wait_for_rereview"
   title="Re-review published content"
   expression="python:object.Modified() + 3*30"
   continue_activity="review" />
\end{verbatim}
 
%%%%%%%%%%%%%%%%%%%%%%%%%%%%%%%%%%%%%%%%%%%%%%%%%%%%%%%%%%%%%%%%%%%%%%%%%%%%%%%
\section{Other Activities}
%%%%%%%%%%%%%%%%%%%%%%%%%%%%%%%%%%%%%%%%%%%%%%%%%%%%%%%%%%%%%%%%%%%%%%%%%%%%%%%

A small amount of activities do not fit well into the previous categories.

%%%%%%%%%%%%%%%%%%%%%%%%%%%%%%%%%%%%%%%%%%%%%%%%%%%%%%%%%%%%%%%%%%%%%%%%%%%%%%%
\subsection{\module{Route} --- Complex split and join operations}
\declaremodule{standard}{route}
\modulesynopsis{Complex split and join operations}
 
The \module{route} activity is a comprehensive and powerful implementation to manage complex
split and join operations within a workflow. As parallel splits are a natural
part of AlphaFlow and other kinds of splits are implemented by the individual
activities, the \module{route} activity provides the tools for joining those
threads again.

A route is a kind of entry point that starts multiple activities and provides
one or more defined ``gates'' to join those routes.

When a gate is reached, it registers this. Depending on the mode of the gate,
it will start its \var{continue\_activity} and either wait for more routes to
trigger it or, when completed, will kill all remaining routes and gates.

For example, routes are used for:

\begin{itemize}
    \item multiple parallel reviews
    \item the milestone pattern (see the ``Recipes'' section)
    \item XOR operations (either one task has to be completed or another)
\end{itemize}

\subsubsection{Activity configuration}

Some rules about the configuration of a route:

\begin{itemize}
    \item The activities listed within a route are the individual routes that are started when the route is started.
    \item Those routes are called ``competing'' routes and are identified by their initial activity.
    \item Routes can refer to activities outside the route activity.
    \item \var{gate}s are given in parallel to the routes. They are not routes themselves.
    \item Gates automatically know which routes exist and which activities belong to those routes.
    \item Gates are triggered by checking which competing route the activity that triggered it belongs to.
    \item Each competing route can trigger a gate only once.
    \item When the first gate is completed, all other gates will be terminated. Every running activity 
        from the remaining competing routes and the route activity itself will be terminated as well.
\end{itemize}
        
\begin{funcdesc}{gate}{id, mode, continue\_activity}
    Specifies a gate within the route. Can be given multiple times. When the
    gate is triggered, the continue\_activity will be started. When the gate
    completes, other gates and the remaining routes will be killed.

    The mode determines when the gate triggers and completes:

    \begin{description}

        \item[multi-merge] Triggers once for every route when it reaches the
        gate. Completes when all routes have triggered the gate.

        \item[discriminate] Triggers when the first route reaches the gate.
        Completes directly after triggering.

        \item[delayed-discriminate] Triggers when the last route is about to be
        completed and at least one route previously reached this gate.
        Completes directly after triggering.

        \item[synchronizing-merge] Triggers when all routes have reached the
        gate. Completes directly after triggering.

    \end{description}
\end{funcdesc}

\begin{notice}
    A smart combination of gates in various modes is very powerful.  Check the
    examples to get a good overview how various combinations work.
\end{notice}

\begin{notice}
    The delayed-discriminate mode allows to implement a review where everybody
    has to give his decision although the result is already known.
\end{notice}

\subsubsection{Instance configuration}

Not applicable.

\subsubsection{ALF example}

This example implements a multi review: different decisions have either be all
together be accepted or they are rejected all. Note: The graphical
representation of the multi-review is very systematic and appealing!

\begin{verbatim}
<route id="demo_route">
    <decision id="decision_1"
        accept_activity="accept" 
        reject_activity="reject" 
        .../>
    <decision id="decision_2"
        accept_activity="accept" 
        reject_activity="reject" 
        .../>
    <decision id="decision_3"
        accept_activity="accept" 
        reject_activity="reject" 
        .../>

    <gate id="reject" mode="discriminate" continue="everything_was_rejected"/>
    <gate id="accept" mode="synchronizing-merge" continue="everything_was_rejected"/>
</route>
\end{verbatim}

See the milestone pattern for another very powerful application of the \module{route} activity.

%%%%%%%%%%%%%%%%%%%%%%%%%%%%%%%%%%%%%%%%%%%%%%%%%%%%%%%%%%%%%%%%%%%%%%%%%%%%%%%
\subsection{\module{Recursion} --- Apply workflow to many objects at once}

\declaremodule{standard}{Recursion}
\modulesynopsis{Apply workflow to many objects at once}

When managing content with a workflow it is sometimes very convenient to
publish a complete folder with all contained objects at once. In Zope and Plone
this requires you to modify status labels, permissions and more for a complete
hierarchy of objects.

In AlphaFlow you do not have to manage a workflow instance for each of those
objects but control them from a central workflow that is run on the top-most
folder.

The \module{recursion} activity duplicates the workflow for certain activities
for every contained object and manages the suppression of parts of the
workflow. In theory you could replicate e.g. review decisions for each single
sub-object, but in practice you only want some parts of the workflow to be
replicated.

The \module{recursion} is one of the few daemon work items: it has to be
enabled once and will stay active in the workflow until the workflow instance
itself is completed. It then monitors what other activities are started and
duplicates them on the sub-objects, starting little sub-workflows on them.

\subsubsection{Activity configuration}

\begin{memberdesc}{recursion\_activity}
    The activities that should be applied recursively. They are started on all sub-objects
    and will run their following activities as well, resulting in sub-workflows for the sub-objects.
\end{memberdesc}
\begin{memberdesc}{break\_activity}
    To avoid completely duplicated workflows on the sub-objects, the activities listed as \var{break\_activity}
    will be suppressed when started on a sub-object.
\end{memberdesc}
\begin{memberdesc}{optional\_recursion}
    If the content object is a folder and this is set to \var{True}, the user who configures the
    workflow will be allowed to choose whether the workflow should only be applied to the
    content object or to the sub-objects as well.
    If this is set to \var{False} the recursion will always be active if the content object
    is folderish.

    Typically an archive workflow would have mandatory recursion as you wouldn't archive a folder
    but not it's contents.
\end{memberdesc}

\subsubsection{Instance configuration}

    \begin{description}
        \item[apply] Lets the user select whether the workflow should be applied 
            recursively or not.
    \end{description}

\subsubsection{ALF example}

This example asks an author to prepare a folder for publication. Notice that
the recursion will trigger the ``perm\_pending'' and ``dc\_pending'' activities
but not the ``review'' activity.

\begin{verbatim}
<workflow ...
    startActivity="prepare_folder recurse">
    
    <recursion id="recurse"
        recursion_activity="perm_pending"
        break_activity="review"
        optional="False"/>
        
    <task id="prepare_folder"
          completion_activity="perm_pending">
          ...
    </task>

    <permission id="perm_pending"
        continue_activity="dc_pending review">
        <add name="Modify portal content" roles="Reviewer"/>
        <remove name="Modify portal content" roles="Owner"/>
    </permission>

    <dcworkflow id="dc_pending" state="pending"/>
    
    <decision id="review" .../>

    ...
\end{verbatim}

